%!TEX root = ../main.tex
\section{Collection of (not so) obvious advice}
\label{sec:advice}

This is a collection of common wisdom that applies to building ML algorithms and neural networks in particular.
The list is definitely not exhaustive, and mostly gathered by surfing Twitter \cite{twitter:karpathy:nn:300618}, asking gazillion question on StackOverflow \cite{se:stats:sycorax:nn-no-learn}, going through blogs \cite{byu:pcc:advice} and through countless sleepless nights.

\subsection{For all networks}

\subsubsection*{If it doesn't work -- try something else.}
Fundamentally, neural networks take lots of {\bf experimentation} to get things to work.
If what you are doing is not getting you what you want, consider going back to the drawing board or make some changes.
But before that perform some basic checks:
\begin{itemize}
\item Write unittests; check for obvious bugs
\item Check your initialization of the weights
\item Make sure your learning rate is not too high / too low
\end{itemize}

\subsubsection*{Keep a logbook of what you are doing!}
% This one bit me so many times, and I still keep on making this mistake: a really cool idea comes to mind and you spend several hours implementing it.
% Somewhere around 3am you finally get the results which show no improvement.
% At that moment, you also remember that you ran a similar experiment several weeks ago -- what a bummer!

An ideal log-book would have information about the architecture, how many epochs it took to overfit/converge, accuracy and loss (training/validation/test), and any notes for your future self.
I know this is tedious, but trust me, being organized that way pays off.

% Don't hard-code your network.
% Instead, create a configuration file that is read and used to populate all the details at run-time.
% Don't delete any of the configuration files -- even if you think the changes are ``insignificant'', create a new configuration file.
% Also, keep track of all of the per-epoch losses for training and validation within the configuration file (e.g. comments).

This serves multiple purposes: 1) Prevents you from making the same mistake; 2) Allows you to review past experiments; 3) Let's you track the progress and dynamics of the project.

% \subsubsection*{Do not (fully) train your network if you can avoid it!}
% There are a lot of architectures on the market right now that are already pre-trained and ready to go.
% Why not using those?
% % Suppose you have a cool new idea -- classifying which lane on 101 freeway would a vehicle take.
% Instead of building a network and training it from scratch -- why not taking an existing pre-trained network (i.e. VGG-16 \missingcitation) and using it as a feature extractor?
% I know, I know, this is what they always teach us -- but why does everyone start from scratch every time then?

\subsubsection*{My view on losses.}
\begin{itemize}
\item {\em Training} loss {\bf stuck} or {\bf oscillating}: Change the learning rate.
\item {\em Training loss} {\bf going down} while {\em validation loss} is {\bf stuck}: the model is overfitting. Add regularization or more data.
\item Both {\em Training} and {\em Validation} loss are {\bf low}, but the {\em Test} loss is {\bf high}: Model is overfitting the validation set. Change the validation set, reshuffle the data, use K-fold validation, or nuke your model and start anew :)
\end{itemize}

\subsubsection*{First train with all-zero data.}
If you start by training on all-zero data, you can catch simple wiring bugs early on.
In addition to that, if you zero data training produces nice, smooth loss curve -- your initialization is wrong.

\subsubsection*{Start with a small network and a small batch of data.}
Before improving your model, try overfitting a single batch first.
In fact, grow your model with your data.
Create a baseline model, overfit it using small batch, keep adding more data until you start under-fitting, make model more complex, rinse and repeat.

Another problem associated with starting with complex models is long turn-arounds.
Imagine you have a model with an SSD bounding box detector that crops images and pushes them into an LSTM to combine everything for tracking.
Just initializing such a model would take 5-10 minutes on a GPU (I got this from \cite{twitter:karpathy:nn:300618} and actually tested it).
Prototyping such a model is a nightmare!!!

\subsubsection*{Make sure that your initial classification loss starts at $\ln N_\text{classes}$}
Remember, if your model is not trained, its log-loss should always be $\ln N_\text{classes}$.
This is a simple math and it works no matter what your real distribution of the data is:
\begin{align*}
L_\text{log} &= \frac{1}{N}\sum_{n=1}^{N}\sum_{k=1}^{K}-y_k^{(n)}\ln\hat{y}_k^{(n)} \\
\sum_{k=1}^{K}y_k^{(\cdot)} &= 1 \\
\text{Initially } \hat{y}_i^{(\cdot)} = \hat{y}_j^{(\cdot)} &= 1/K, \forall{i,j}\\
\Rightarrow L_\text{log}^\text{initial} &= \ln{K}
\end{align*}

\subsubsection*{Check if your loss gets correct feed.}
Don't route \verb+softmax+ outputs to a loss that expects raw logits.

\subsubsection*{Design-for-Debug: Generalize your model well.}
Make sure you design your system with debugging in mind.
For example, if you have a recurrent network, and you only need a single prediction, just output predictions at each step.
That way you can debug it.
Without generalizing your model it is very hard to find intermediate issues.
For some networks you would need to look deep inside their layers -- I guess for simple networks you TensorBoard can help \cite{tensorflow2015-whitepaper}.

\subsubsection*{Tripple-check your data augmentation.}
This is actually about all pre-processing.
\begin{itemize}
\item Make sure you augment the copy of the image not the image itself;
\item Make sure your augmentation is reasonable and realistic;
\item If labels depend on orientation -- don't forget to change the labels as well. I made this mistake when I was training the turn angle using a camera feed. Was augmenting the data by using horizontal flips, but forgot to change the steering angle *facepalm*.
\end{itemize}

%%%%%%%%%%%%%%%%%

\subsubsection*{Overfitting / Underfitting}
No need to explain that -- just keep track of your trainin/validation losses.
Often times tuning the hyperparameters is a process when you constantly look at the loss plots :)

\subsubsection*{Dropout, Normalization, and Clipping}
Use dropout, but not for the convolutional layers!
Usually the dropout with the 50\% rate is good enough, but feel free to experiment :)

Use batch normalization -- this reduces the covariance shift, and allows us to
\begin{itemize}
\item Use higher learning rates because batch normalization makes sure that there’s no activation that’s gone really high or really low.
\item Slightly reduce overfitting because it has a slight regularization effects.
\end{itemize}

In addition to tracking your losses, you also might want to track your gradients.
If you notice that the gradients are overshooting, it is often useful to clip them.
Generally, applying the \texttt{L2 Norm Clipping} before applying them with your favorite optimizer helps with the {\bf Gradient Explosion Problem}.

\subsection{For Generative Adversarial Networks}
\subsubsection*{Activation Functions and Pooling}
Avoid sparse gradients: instead of \verb+ReLU+ you should probably use \verb+LeakyReLU+.

Same goes for the pooling -- \verb+MaxPool+ is the best, except the places where it's not: \verb+AveragePool+ is much better for the architectures which don't like sparse gradients.

\subsubsection*{Normalization}
It was shown that when you are trainign your discriminator, you should normalize the "real" and "fake" (generated) batches separately.
Don't mix the "real" and "fake" data into the same batch.

\subsubsection*{Optimizers}
Use different optimization algorithms for your generator and your discriminator.
For example you can use \verb+Adam+ for the generator, and \verb+SGD+ for the discriminator.

\subsubsection*{Loss and Gradients}
\begin{itemize}
\item[\color{red!80!black}BAD] Discriminator loss is approaching zero.
\item[\color{red!80!black}BAD] Gradient norms are oscillating or blowing to infinity.
\item[\color{red!80!black}BAD] Generator loss is steadily going down (discriminator is being fooled!!!)
\item[\color{green!60!black}GOOD] Discriminator loss has low variance and slowly going down.
\end{itemize}

\subsubsection*{Generator vs. Discriminator Ratio}
So far I didn't see anyone being able to answer conclusively -- what is the balance between the generator and discriminator usage during training.

\subsection{For Reinforcement Learning}
Just use the {\bf Rainbow}\cite{hessel:arxiv:2017}
\zafar{This section is a stub}

\subsection{For Recurrent Networks}

\subsubsection*{Don't shuffle your data}
Everyone forgets about this one.
You cannot shuffle the data when working with the recurrent networks (unless you know what you are doing).

\subsubsection*{Clip your gradients}
Although this is optional for generic networks, for RNNs you most probably need to clip your gradients.

\subsubsection*{Normalize your loss}
The loss should be averaged across the batch.
Sum up your loss, and nomalize it by the sequence length.
This will allow the losses to be of comparable magnitude.

\subsubsection*{Use feed-forward as the first layer}
Although that's not what you usually see, it is worth trying to add a feed-forward layer or two first.
This will transform the input sequence into more ``digestable'' dimensions.

\subsubsection*{Use ``ResNet''-like architectures}
Recurrent networks need a lot of parameters.
To resolve it, stack several smaller recurrent layers and sum the outputs of multiple layers in the end (just like ResNet).

